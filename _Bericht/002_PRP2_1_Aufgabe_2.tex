\chapter*{Aufgabe 2: Schach}

\textbf{(1) Aufbau eines Arrays, dass das Schachbrett und die Schachfiguren darstellen
kann}

Es wurde ein zweidimensionales char-Array (playground) initialisiert. Die Größe des Arrays ist über ein Makro (SIZE) definiert. In unserem Fall entspricht die Größe den Dimensionen eines Schachfeldes.
Bei dem Feld A1 fängt das Array bei [0][0] an. \\
Weiße Felder werden durch ein Leerzeichen dargestellt, für schwarze wird der Ascii-Code 177 verwendet.\\

\textbf{(2) String der Schachnotation in einzelne Züge aufschlüsseln}

Wurde mit \textbf{strtok} (string token) realisiert, durch Festlegen eines Trennzeichens (hier: „/“). Rückgabe ist ein Pointer, der auf die einzelnen gespaltenen Teile zeigt. In einer Schleife wird dieser Prozess wiederholt, bis alle Kommandos abgearbeitet wurden.\\

\textbf{(3) Automatisches Nachspielen der gegebenen Züge}

Nachdem ein Kommando aus dem Partie-String extrahiert wurde, wird die Funktion \textbf{play-str} aufgerufen und die Länge des Kommandos abgefragt. Bei einer Länge von 5 ist keine Spielfigur angegeben und es muss sich um einen Bauern handeln. Bei einer Länge von 6 ist die Spielfigur festgelegt.
Als nächstes wird überprüft, ob es sich um einen Setzzug oder Schlagzug handelt. Die Stelle des ausschlaggebenden Characters ist durch die bereits abgeschlossene Abfrage zur Länge des Kommandos eindeutig. \\
Durch die counter-Variable, die in jedem Schleifendurchlauf erhöht wird, kann überprüft werden, welcher Spieler gerade am Zug ist. Mit der Modulo-Operation wird überprüft, ob die Spielfigur als Groß- oder Kleinbuchstabe in das playground-Array gespeichert werden muss.
Anschließend wird die Funktion \textbf{update-playgroud} aufgerufen und das Spielfigur wird im Array auf die Zielposition gesetzt.
Das Startfeld wird entweder durch ein weißes oder schwarzes Feld ersetzt.\\

\textbf{(4) Darstellung des veränderten Schachbrettes auf dem Bildschirm}

Ausgegeben wird das Spielfeld mit einer einfachen for-Schleife in der Funktion \textbf{print-playground}.\\

\textbf{(5,6) Benutzer soll das Spiel weiterspielen}

Sobald der Partie-String vollständig gespielt wurde, wird in der main-Funktion gefragt, ob der Spieler weiterspielen möchte oder nicht. Dies kann mit einer einfachen Eingabe von "j" bestätigt und mit "n" verneint werden. Falls die Frage mit einem "Nein" beantwortet wurde, wird das Programm beendet. Falls weitergespielt werden möchte, wird die Funktion \textbf{play-input} aufgerufen.
Hier wird zunächst die Startposition und die Zielposition abgefragt. Bei der Eingabe von "00" als Startpostion wird das Spiel beendet.
Als nächstes werden die x- und y-Koordinaten der Startposition in Integer umgewandelt und die Figur, welche sich auf der Startposition befindet herausgearbeitet. Anschließend wird die Funktion \textbf{update-playground} aufgerufen und das Spielfeld wie bereits in (3) nachgespielt und anschließend auf der Konsole mit der Funktion \textbf{print-playground} ausgegeben.


